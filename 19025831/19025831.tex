\documentclass[11pt,preprint, authoryear]{elsarticle}

\usepackage{lmodern}
%%%% My spacing
\usepackage{setspace}
\setstretch{1.2}
\DeclareMathSizes{12}{14}{10}{10}

% Wrap around which gives all figures included the [H] command, or places it "here". This can be tedious to code in Rmarkdown.
\usepackage{float}
\let\origfigure\figure
\let\endorigfigure\endfigure
\renewenvironment{figure}[1][2] {
    \expandafter\origfigure\expandafter[H]
} {
    \endorigfigure
}

\let\origtable\table
\let\endorigtable\endtable
\renewenvironment{table}[1][2] {
    \expandafter\origtable\expandafter[H]
} {
    \endorigtable
}


\usepackage{ifxetex,ifluatex}
\usepackage{fixltx2e} % provides \textsubscript
\ifnum 0\ifxetex 1\fi\ifluatex 1\fi=0 % if pdftex
  \usepackage[T1]{fontenc}
  \usepackage[utf8]{inputenc}
\else % if luatex or xelatex
  \ifxetex
    \usepackage{mathspec}
    \usepackage{xltxtra,xunicode}
  \else
    \usepackage{fontspec}
  \fi
  \defaultfontfeatures{Mapping=tex-text,Scale=MatchLowercase}
  \newcommand{\euro}{€}
\fi

\usepackage{amssymb, amsmath, amsthm, amsfonts}

\def\bibsection{\section*{References}} %%% Make "References" appear before bibliography


\usepackage[round]{natbib}

\usepackage{longtable}
\usepackage[margin=2.3cm,bottom=2cm,top=2.5cm, includefoot]{geometry}
\usepackage{fancyhdr}
\usepackage[bottom, hang, flushmargin]{footmisc}
\usepackage{graphicx}
\numberwithin{equation}{section}
\numberwithin{figure}{section}
\numberwithin{table}{section}
\setlength{\parindent}{0cm}
\setlength{\parskip}{1.3ex plus 0.5ex minus 0.3ex}
\usepackage{textcomp}
\renewcommand{\headrulewidth}{0.2pt}
\renewcommand{\footrulewidth}{0.3pt}

\usepackage{array}
\newcolumntype{x}[1]{>{\centering\arraybackslash\hspace{0pt}}p{#1}}

%%%%  Remove the "preprint submitted to" part. Don't worry about this either, it just looks better without it:
\makeatletter
\def\ps@pprintTitle{%
  \let\@oddhead\@empty
  \let\@evenhead\@empty
  \let\@oddfoot\@empty
  \let\@evenfoot\@oddfoot
}
\makeatother

 \def\tightlist{} % This allows for subbullets!

\usepackage{hyperref}
\hypersetup{breaklinks=true,
            bookmarks=true,
            colorlinks=true,
            citecolor=blue,
            urlcolor=blue,
            linkcolor=blue,
            pdfborder={0 0 0}}


% The following packages allow huxtable to work:
\usepackage{siunitx}
\usepackage{multirow}
\usepackage{hhline}
\usepackage{calc}
\usepackage{tabularx}
\usepackage{booktabs}
\usepackage{caption}


\newenvironment{columns}[1][]{}{}

\newenvironment{column}[1]{\begin{minipage}{#1}\ignorespaces}{%
\end{minipage}
\ifhmode\unskip\fi
\aftergroup\useignorespacesandallpars}

\def\useignorespacesandallpars#1\ignorespaces\fi{%
#1\fi\ignorespacesandallpars}

\makeatletter
\def\ignorespacesandallpars{%
  \@ifnextchar\par
    {\expandafter\ignorespacesandallpars\@gobble}%
    {}%
}
\makeatother

\newlength{\cslhangindent}
\setlength{\cslhangindent}{1.5em}
\newenvironment{CSLReferences}%
  {\setlength{\parindent}{0pt}%
  \everypar{\setlength{\hangindent}{\cslhangindent}}\ignorespaces}%
  {\par}


\urlstyle{same}  % don't use monospace font for urls
\setlength{\parindent}{0pt}
\setlength{\parskip}{6pt plus 2pt minus 1pt}
\setlength{\emergencystretch}{3em}  % prevent overfull lines
\setcounter{secnumdepth}{5}

%%% Use protect on footnotes to avoid problems with footnotes in titles
\let\rmarkdownfootnote\footnote%
\def\footnote{\protect\rmarkdownfootnote}
\IfFileExists{upquote.sty}{\usepackage{upquote}}{}

%%% Include extra packages specified by user
\usepackage{tablefootnote}

%%% Hard setting column skips for reports - this ensures greater consistency and control over the length settings in the document.
%% page layout
%% paragraphs
\setlength{\baselineskip}{12pt plus 0pt minus 0pt}
\setlength{\parskip}{12pt plus 0pt minus 0pt}
\setlength{\parindent}{0pt plus 0pt minus 0pt}
%% floats
\setlength{\floatsep}{12pt plus 0 pt minus 0pt}
\setlength{\textfloatsep}{20pt plus 0pt minus 0pt}
\setlength{\intextsep}{14pt plus 0pt minus 0pt}
\setlength{\dbltextfloatsep}{20pt plus 0pt minus 0pt}
\setlength{\dblfloatsep}{14pt plus 0pt minus 0pt}
%% maths
\setlength{\abovedisplayskip}{12pt plus 0pt minus 0pt}
\setlength{\belowdisplayskip}{12pt plus 0pt minus 0pt}
%% lists
\setlength{\topsep}{10pt plus 0pt minus 0pt}
\setlength{\partopsep}{3pt plus 0pt minus 0pt}
\setlength{\itemsep}{5pt plus 0pt minus 0pt}
\setlength{\labelsep}{8mm plus 0mm minus 0mm}
\setlength{\parsep}{\the\parskip}
\setlength{\listparindent}{\the\parindent}
%% verbatim
\setlength{\fboxsep}{5pt plus 0pt minus 0pt}



\begin{document}



\begin{frontmatter}  %

\title{FTSE/JSE Listed Property: Investigating Time-Varying Correlations
Using a DCC MV-GARCH Model}

% Set to FALSE if wanting to remove title (for submission)




\author[Add1]{Tian Cater\footnote{\emph{This project was generated using
  Katzke (\protect\hyperlink{ref-Texevier}{2017}), a package to create
  Elsevier templates for Rmarkdown.}}}
\ead{19025831@sun.ac.za}





\address[Add1]{Financial Econometrics 871 Project 2022}
\address[Add2]{University of Stellenbosch, Western Cape, South Africa}



\vspace{1cm}





\vspace{0.5cm}

\end{frontmatter}



%________________________
% Header and Footers
%%%%%%%%%%%%%%%%%%%%%%%%%%%%%%%%%
\pagestyle{fancy}
\chead{}
\rhead{}
\lfoot{}
\rfoot{\footnotesize Page \thepage}
\lhead{}
%\rfoot{\footnotesize Page \thepage } % "e.g. Page 2"
\cfoot{}

%\setlength\headheight{30pt}
%%%%%%%%%%%%%%%%%%%%%%%%%%%%%%%%%
%________________________

\headsep 35pt % So that header does not go over title




\hypertarget{introduction}{%
\section{\texorpdfstring{Introduction
\label{Introduction}}{Introduction }}\label{introduction}}

Investors tend to diversify components of their investment portfolios to
property assets to mitigate their portfolio's downside risk. In
addition, listed property securities are attractive as they generally
yield high dividends and showcase potential moderate long-term capital
appreciation. However, in the past decade, FTSE/JSE-listed property
securities have performed weakly and have become significantly more
volatile. Moreover, the correlation between FTSE/JSE-listed property
securities and alternative financial assets has become more positive,
placing its diversification benefits under scrutiny
(\protect\hyperlink{ref-carstens2020pull}{Carstens \& Freybote, 2020};
\protect\hyperlink{ref-ijasan2017anti}{Ijasan, Tweneboah \& Mensah,
2017}; \protect\hyperlink{ref-oberholzer2015univariate}{Oberholzer \&
Venter, 2015}).

To this extent, this project investigates time-varying conditional
correlations between FTSE/JSE-listed property securities and the broader
FTSE/JSE assets by adopting the parsimonious Dynamic Conditional
Correlation (DCC) Multivariate Generalized Autoregressive Conditional
Heteroskedasticity (MV GARCH) modelling procedure.

To do so, I construct a market capitalisation-weighted portfolio of all
the FTSE/JSE-listed property constituents, dubbed PROP, and compare its
time-varying conditional correlations with the JSE Small Cap Index
(SMLC(J202)) and Shareholder Weighted Index (SWIX(J433)). The
self-constructed PROP index closely imitates the JSE The FTSE/JSE All
Property Index (J803), however, also includes all small-and-medium-cap
constituents. The indexes' cumulative returns, together with the JSE Top
40 Index (Top40(J200)), are shown in Figure \ref{Figure1}
below.\footnote{All indexes' weighting is capped at 10\(\%\), except for
  the SMLC(J202), which is capped at 15\(\%\).} The PROP index
performance has weakened and becomes increasingly unstable in the recent
decade, amplified during recessionary periods (blue-shaded area).

\begin{figure}[H]

{\centering \includegraphics{19025831_files/figure-latex/Figure1-1} 

}

\caption{Cumulative Returns \label{Figure1}}\label{fig:Figure1}
\end{figure}

The DCC MV-GARCH model's results suggest that, on aggregate, the
comovement between PROP, SWIX(J433), and SMLC(J202) is amplified during
periods of heightened global economic uncertainty. Although the PROP
index is the most volatile of the indexes and the SMLC(J202) index the
least volatile, it exhibits a substantially lower time-varying
conditional correlation with the SWIX(J433) index compared to the
SMLC(J202) index. However, since 2016, this difference has shrunk
marginally. On the other hand, the dynamic conditional correlation
between the SMLC(J202) and SWIX(J433) indexes is the highest and the
more stable among the three index pairs considered. These results infer
that an investor holding large proportions of SWIX(J433) constituents
will achieve superior portfolio diversification in purchasing listed
property compared to SMLC(J202) constituents.

\hypertarget{results-inter-index-comovement-and-estimated-volatility}{%
\section{Results: Inter-Index Comovement and Estimated
Volatility}\label{results-inter-index-comovement-and-estimated-volatility}}

The time-varying dynamic conditional correlations (DCC) are estimated
using the estimated univariate GARCH(1,1) models' standardised residuals
in the second stage of estimation.

The estimated volatility for each index and time-varying DCC between the
PROP, SMLC(J202), and SWIX(J433) indexes are depicted in Figures
\ref{Figure2} and \ref{Figure3}, respectively.\footnote{Also see
  \ref{aa} for the (unmodeled) sample scaled growth and standard
  deviation in Figures \ref{aa1} and \ref{aa2}, respectively.} Figure
\ref{Figure3} shows that heterogeneity exists between the index pairs
over time, indicating that stationary correlation modelling estimates
(for example, the Constant Conditional Correlations or CCC) could be
deceptive.\footnote{I drop the Top40(J200) index due to its similar
  dynamics to the SWIX(J433) index.}

Table \ref{DCC} reports the coefficients \(a\) and \(b\)'s estimates and
corresponding p-values. From Katzke \& others
(\protect\hyperlink{ref-katzke2013}{2013}), these estimates signify mean
reversion of the time-varying correlations since \(a + b < 1\). The
impact of lagged standardised shocks on dynamic conditional correlations
is measured by the coefficient \(a\). In contrast, the measure of the
past effect of the dynamic conditional correlations on present dynamic
conditional correlations is given by \(b\). These parameters are,
additionally, statistically significant at the 5\(\%\) level, except for
the \(a\) and \(b\) coefficients for SMLC(J202), which is significant at
the 10\(\%\) level. Again following Katzke \emph{et al.}
(\protect\hyperlink{ref-katzke2013}{2013}), this indicates significant
deviations over time, reaffirming that a DCC model is more fitting than
a CCC model.

The corresponding estimated DCC model diagnostics, checking for
conditional heteroscedasticity through testing for serial correlation,
is reported in Table \ref{MARCH} below. As stated by Tsay
(\protect\hyperlink{ref-tsay2013}{2013}), when the shocks are
heavy-tailed, the parameters \(Q(m)\) and \(Q_k(m)\) often fail to
detect the presence of conditional heteroscedasticity, and the
\(Q_k^r (m)\) robustness parameter is desirable. Consequently, the
fitted DCC model fails to reject the null of no autocorrelation when
considering the rank-based test and the robustness parameter
\(Q_k^r (m)\).

The model's estimated volatility for each index (Figure \ref{Figure2})
shows that PROP is the most volatile in comparison to the SMLC(J202) and
SWIX(J433) indexes, especially in the past decade, possessing
substantially larger jumps in volatility during recessionary periods
(blue shaded areas). Moreover, the SMLC(J202) index is the least
volatile.

In analysing the estimated dynamic conditional correlations across index
pairs (Figure \ref{Figure3}), the PROP index exhibits a substantially
lower time-varying conditional correlation with the SWIX(J433) index
compared to the SMLC(J202) index. However, since 2016, this difference
has shrunk marginally. On the other hand, the dynamic conditional
correlation between the SMLC(J202) and SWIX(J433) indexes is the highest
and the more stable among the three index pairs considered.

\begin{figure}[H]

{\centering \includegraphics{19025831_files/figure-latex/Figure2-1} 

}

\caption{DCC GARCH: Estimated Volatility \label{Figure2}}\label{fig:Figure2}
\end{figure}

\begin{figure}[H]

{\centering \includegraphics{19025831_files/figure-latex/Figure3-1} 

}

\caption{DCC GARCH: Dynamic Conditional Correlations \label{Figure3}}\label{fig:Figure3}
\end{figure}

\begin{center}
\begin{longtable}{|ccc|}
\caption{DCC Model \label{DCC}} \\
\hline
\multicolumn{1}{|c}{}& 
\multicolumn{1}{c}{$a$}& 
\multicolumn{1}{c|}{$b$}\\
\hline \hline
\endhead
SWIX(J433)  & 0.147 & 0.841 \\ 
            & (0.0345) & (0)\\
SMLC(J202)  & 0.135 & 0.836 \\ 
            & (0.09172) & (0.01)\\
PROP        & 0.0996 & 0.8814 \\ 
            & (0.00038) & (0)\\
\hline
\multicolumn{3}{|l|}{Note: P-values given in brackets}\\
\hline
\end{longtable}
\end{center}

\begin{center}
\begin{longtable}{|cccc|}
\caption{Model Diagnostics \label{MARCH}} \\
\hline
\multicolumn{1}{c}{$Q(m)$}& 
\multicolumn{1}{c}{$Rank-based \ test$}& 
\multicolumn{1}{c}{$Q_k(m)$}&
\multicolumn{1}{c|}{$Q_r^{k} (m)$}\\
\hline \hline
\endhead
599.07  & 12.69 & 369.79 & 175.44\\ 
(0)    & (0.25044) & (0.0784 & (0.19111)\\
\hline
\multicolumn{4}{|l|}{Note: P-values given in brackets.}\\
\hline
\end{longtable}
\end{center}

\hypertarget{conclusion}{%
\section{Conclusion}\label{conclusion}}

The purpose of this project is to investigate time-varying conditional
correlations between FTSE/JSE-listed property securities and the broader
FTSE/JSE assets by adopting the Dynamic Conditional Correlation (DCC)
Multivariate Generalized Autoregressive Conditional Heteroskedasticity
(MV GARCH) modelling procedure.

The DCC MV-GARCH model's results suggest that, on aggregate, the
comovement between PROP, SWIX(J433), and SMLC(J202) is amplified during
periods of heightened global economic uncertainty. Although the PROP
index is the most volatile of the indexes and the SMLC(J202) index the
least volatile, it exhibits a substantially lower time-varying
conditional correlation with the SWIX(J433) index compared to the
SMLC(J202) index. However, since 2016, this difference has shrunk
marginally. On the other hand, the dynamic conditional correlation
between the SMLC(J202) and SWIX(J433) indexes is the highest and the
more stable among the three index pairs considered. These results infer
that an investor holding large proportions of SWIX(J433) constituents
will achieve superior portfolio diversification in purchasing listed
property compared to SMLC(J202) constituents.

\newpage

\hypertarget{references}{%
\section*{References}\label{references}}
\addcontentsline{toc}{section}{References}

\hypertarget{refs}{}
\begin{CSLReferences}{1}{0}
\leavevmode\vadjust pre{\hypertarget{ref-carstens2020pull}{}}%
Carstens, R. \& Freybote, J. 2020. Pull and push factors as determinants
of foreign REIT investments. \emph{Journal of Real Estate Portfolio
Management}. 25(2):151--171.

\leavevmode\vadjust pre{\hypertarget{ref-ijasan2017anti}{}}%
Ijasan, K., Tweneboah, G. \& Mensah, J.O. 2017. Anti-persistence and
long-memory behaviour of SAREITs. \emph{Journal of Property Investment
\& Finance}. 35(4):356--368.

\leavevmode\vadjust pre{\hypertarget{ref-katzke2013}{}}%
Katzke, N. et al. 2013. South african sector return correlations: Using
DCC and ADCC multivariate GARCH techniques to uncover the underlying
dynamics. \emph{South African Sector Return Correlations: using DCC and
ADCC Multivariate GARCH techniques to uncover the underlying dynamics}.
10--17.

\leavevmode\vadjust pre{\hypertarget{ref-Texevier}{}}%
Katzke, N.F. 2017. \emph{{Texevier}: {P}ackage to create elsevier
templates for rmarkdown}. Stellenbosch, South Africa: Bureau for
Economic Research.

\leavevmode\vadjust pre{\hypertarget{ref-oberholzer2015univariate}{}}%
Oberholzer, N. \& Venter, P. 2015. Univariate GARCH models applied to
the JSE/FTSE stock indices. \emph{Procedia Economics and Finance}.
24:491--500.

\leavevmode\vadjust pre{\hypertarget{ref-tsay2013}{}}%
Tsay, R.S. 2013. \emph{Multivariate time series analysis: With r and
financial applications}. John Wiley \& Sons.

\end{CSLReferences}

\newpage
\appendix
\renewcommand{\thesection}{Appendix A}

\hypertarget{section}{%
\section{\texorpdfstring{\label{aa}}{}}\label{section}}

\begin{figure}[H]

{\centering \includegraphics{19025831_files/figure-latex/aa1-1} 

}

\caption{Scaled Growth\label{aa1}}\label{fig:aa1}
\end{figure}

\begin{figure}[H]

{\centering \includegraphics{19025831_files/figure-latex/aa2-1} 

}

\caption{Sample Standard Deviation \label{aa2}}\label{fig:aa2}
\end{figure}

\bibliography{Tex/ref}





\end{document}
